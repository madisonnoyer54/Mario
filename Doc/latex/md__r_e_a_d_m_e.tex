I. Présentation

Le jeu a été conçu par N\+O\+Y\+ER Madison et S\+C\+H\+I\+L\+L\+I\+NG Juliette, dans le cadre de l\textquotesingle{}UE de programmation avancé. Le But du jeu est d\textquotesingle{}arriver à la fin du 3eme niveau, sans mourir. On peut perdre de la vie si on touche un champigon ou si on tombe dans le vide. Au debut du jeu on obtient 3 vies que l\textquotesingle{}on récupère à chaque début de niveau. Le but est de ramasser un maximum de pièce durant la partie. Il y a également des murs, pour permettre d\textquotesingle{}attrapper les pièces en hauteur et éviter plus facilement les champignons

II. Exécution du jeu

Pour exécuter le jeu, il suffit d\textquotesingle{}aller dans le dossier correspondant et de taper les commandes suivantes \+: make puis \+: ./main

I\+II. Binome Juliette schilling -\/noyer Madison adresse du git \+:\href{https://gitlab.univ-lorraine.fr/noyer7u/projet-l2-noyer-schilling}{\texttt{ https\+://gitlab.\+univ-\/lorraine.\+fr/noyer7u/projet-\/l2-\/noyer-\/schilling}} 